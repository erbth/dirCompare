% !TeX spellcheck = en_US
\documentclass[a4paper]{article}
\usepackage[hidelinks]{hyperref}
\usepackage{tabularx}

\title{Documentation for dirCompare}
\author{Thomas Erbesdobler}
\date{Last updated: \today}

\setlength{\parskip}{1ex}

\begin{document}
	\maketitle
	\tableofcontents
	
	\section{What is dirCompare?}
	\label{sec:what_is_dircompare}
	
	''dirCompare'' is an open source tool for comparing directory trees. It is written in C++ and designed to be multi-platform with the capability to add more supported platforms quite easily. This is achieved using the abstract factory pattern.
	
	No initial release has been done yet, furthermore licensing information has still to be added.
	
	\section{User documentation}
	\label{sec:user_documentation}
	
	\subsection{Command line Parameters}
	\label{subsec:command_line_parameters}
	
	The following table lists all available command line parameters and their semantics:
	
	\noindent
	\begin{tabularx}{\linewidth}{l|X}
		\texttt{--dir1 \textless path\textgreater} & Path to first directory for comparison \\
		\texttt{--dir2 \textless path\textgreater} & Path to second directory for comparison \\
		\texttt{--listStrategies} & List file and directory comparison strategies \\
		\texttt{--fileStrategy \textless id\textgreater} & Specify the comparison strategy for files \\
		\texttt{--dirStrategy \textless id\textgreater} & Specify the comparison strategy for directories \\
		\texttt{--logfile \textless path\textgreater} & Use the specified logfile. If none is given, the output is written to \texttt{stdout} or \texttt{stderr} \\
	\end{tabularx}
	
	\subsection{Return code}
	\label{subsec:return_code}
	
	A return code of 0 signals an equality of the two directories in respect to the given comparison strategies. A different return code signals either their inequality or an error/failure like an invalid command line.
	
\end{document}
